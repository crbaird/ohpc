\documentclass[letterpaper]{article}
\usepackage{common/ohpc-doc}
\setcounter{secnumdepth}{5}
\setcounter{tocdepth}{5}

% Include git variables
\input{vc.tex}

% Define Base OS and other local macros
\newcommand{\baseOS}{Rocky 8.8}
\newcommand{\OSRepo}{Rocky\_8.8}
\newcommand{\OSTree}{EL\_8}
\newcommand{\OSTag}{el8}
\newcommand{\baseos}{rocky8.8}
\newcommand{\baseosshort}{rocky8}
\newcommand{\provisioner}{Warewulf}
\newcommand{\provheader}{\provisioner{}}
\newcommand{\rms}{OpenPBS}
\newcommand{\rmsshort}{OpenPBS}
\newcommand{\arch}{x86\_64}

% Define package manager commands
\newcommand{\pkgmgr}{yum}
\newcommand{\addrepo}{wget -P /etc/yum.repos.d}
\newcommand{\chrootaddrepo}{wget -P \$CHROOT/etc/yum.repos.d}
\newcommand{\clean}{yum clean expire-cache}
\newcommand{\chrootclean}{yum --installroot=\$CHROOT clean expire-cache}
\newcommand{\install}{yum -y install}
\newcommand{\chrootinstall}{yum -y --installroot=\$CHROOT install}
\newcommand{\groupinstall}{yum -y groupinstall}
\newcommand{\groupchrootinstall}{yum -y --installroot=\$CHROOT groupinstall}
\newcommand{\remove}{yum -y remove}
\newcommand{\upgrade}{yum -y upgrade}
\newcommand{\chrootupgrade}{yum -y --installroot=\$CHROOT upgrade}
\newcommand{\tftppkg}{syslinux-tftpboot}
\newcommand{\beegfsrepo}{https://www.beegfs.io/release/beegfs\_7.2.1/dists/beegfs-rhel8.repo}

% boolean for os-specific formatting
\toggletrue{isCentOS}
\togglefalse{isSLURM}
\toggletrue{isx86}
\toggletrue{isWarewulf}
\toggletrue{ispbs}
\toggletrue{isCentOS_ww_pbs_x86}
\toggletrue{isCentOS_x86}

\begin{document}
\graphicspath{{common/figures/}}
\thispagestyle{empty}

% Title Page
\input{common/title}
% Disclaimer
\input{common/legal} 

\newpage
\tableofcontents
\newpage

% Introduction  --------------------------------------------------

\section{Introduction} \label{sec:introduction}
\input{common/install_header}
\input{common/intro} \\

\input{common/base_edition/edition}
\input{common/audience}
\input{common/requirements}
\input{common/inputs}

% begin_ohpc_run
% ohpc_validation_newline
% ohpc_validation_comment Verify OpenHPC repository has been enabled before proceeding
% ohpc_validation_newline
% ohpc_command yum repolist | grep -q OpenHPC
% ohpc_command if [ $? -ne 0 ];then
% ohpc_command    echo "Error: OpenHPC repository must be enabled locally"
% ohpc_command    exit 1
% ohpc_command fi
% end_ohpc_run

% Base Operating System --------------------------------------------

\section{Install Base Operating System (BOS)}
\input{common/bos}

% begin_ohpc_run
% ohpc_validation_newline
% ohpc_validation_comment Disable firewall 
\begin{lstlisting}[language=bash,keywords={}]
[sms](*\#*) systemctl disable firewalld
[sms](*\#*) systemctl stop firewalld
\end{lstlisting}
% end_ohpc_run

% ------------------------------------------------------------------

\section{Install \OHPC{} Components} \label{sec:basic_install}
\input{common/install_ohpc_components_intro.tex}

\subsection{Enable \OHPC{} repository for local use} \label{sec:enable_repo}
\input{common/enable_ohpc_repo}
\input{common/rocky_repos}
\input{common/automation}


\subsection{Add provisioning services on {\em master} node} \label{sec:add_provisioning}
\input{common/install_provisioning_intro}
\input{common/enable_pxe}
\input{common/time}

\subsection{Add resource management services on {\em master} node} \label{sec:add_rm}
\input{common/install_openpbs}

\subsection{Optionally add \InfiniBand{} support services on {\em master} node} \label{sec:add_ofed}
\input{common/ibsupport_sms_centos}

\subsection{Optionally add \OmniPath{} support services on {\em master} node} \label{sec:add_opa}
\input{common/opasupport_sms_centos}

\vspace*{-0.15cm}
\subsection{Complete basic Warewulf setup for {\em master} node} \label{sec:setup_ww}
\input{common/warewulf_setup}
\input{common/warewulf_setup_centos}

\subsection{Define {\em compute} image for provisioning}
\input{common/warewulf_mkchroot_rocky}

\subsubsection{Add \OHPC{} components} \label{sec:add_components}
\input{common/add_to_compute_chroot_intro}

% begin_ohpc_run
% ohpc_validation_comment Add OpenHPC components to compute instance
\begin{lstlisting}[language=bash,literate={-}{-}1,keywords={},upquote=true]
# Add OpenPBS client support
[sms](*\#*) (*\chrootinstall*) openpbs-execution-ohpc
[sms](*\#*) perl -pi -e "s/PBS_SERVER=\S+/PBS_SERVER=${sms_name}/" $CHROOT/etc/pbs.conf
[sms](*\#*) echo "PBS_LEAF_NAME=${sms_name}" >> /etc/pbs.conf
[sms](*\#*) chroot $CHROOT opt/pbs/libexec/pbs_habitat
[sms](*\#*) perl -pi -e "s/\$clienthost \S+/\$clienthost ${sms_name}/" $CHROOT/var/spool/pbs/mom_priv/config
[sms](*\#*) echo "\$usecp *:/home /home" >> $CHROOT/var/spool/pbs/mom_priv/config
[sms](*\#*) chroot $CHROOT systemctl enable pbs

# Add Network Time Protocol (NTP) support
[sms](*\#*) (*\chrootinstall*) chrony
# Identify master host as local NTP server
[sms](*\#*) echo "server ${sms_ip} iburst" >> $CHROOT/etc/chrony.conf

# Add kernel drivers (matching kernel version on SMS node)
[sms](*\#*) (*\chrootinstall*) kernel-`uname -r`

# Include modules user environment
[sms](*\#*) (*\chrootinstall*) lmod-ohpc
\end{lstlisting}
% end_ohpc_run

\subsubsection{Customize system configuration} \label{sec:master_customization}
\input{common/warewulf_chroot_customize_centos}
\input{common/oneapi_mountpoint}
\input{common/restart_nfs}

%\clearpage
\subsubsection{Additional Customization ({\em optional})} \label{sec:addl_customizations}
\input{common/compute_customizations_intro}

\paragraph{Enable \InfiniBand{} drivers}
\input{common/ibsupport_compute_centos.tex}

\paragraph{Enable \OmniPath{} drivers}
\input{common/opasupport_compute_centos.tex}


\paragraph{Increase locked memory limits}
\input{common/memlimits}

\clearpage
\paragraph{Add \beegfs{}} \label{sec:add_beegfs}
\input{common/install_beegfs_client_centos}

\vspace*{.6cm}
\paragraph{Add \Lustre{} client} \label{sec:lustre_client}
\input{common/lustre-client}
\input{common/lustre-client-centos}
\input{common/lustre-client-post}

\paragraph{Enable forwarding of system logs} \label{sec:add_syslog}
\input{common/syslog}

\paragraph{Add \Nagios{} monitoring} \label{sec:add_nagios}
\input{common/nagios}

\clearpage
\paragraph{Add \clustershell{}}
\input{common/clustershell}

%\clearpage
\paragraph{Add \genders{}}
\input{common/genders}

\paragraph{Add Magpie}
\input{common/magpie}

\paragraph{Add \conman{}} \label{sec:add_conman}
\input{common/conman}

\vspace*{.25cm}
\paragraph{Add \nhc{}} \label{sec:add_nhc}
\input{common/nhc}

\vspace*{.25cm}
\subsubsection{Import files} \label{sec:file_import}
\input{common/import_ww_files}
\input{common/import_ww_files_ib_centos}
\vspace*{.7cm}
\input{common/finalize_provisioning}
\input{common/add_ww_hosts_intro}
\input{common/add_ww_hosts_pbs}
\clearpage
\input{common/add_ww_hosts_finalize}

%\clearpage
\subsubsection{Optional kernel arguments} \label{sec:optional_kargs}
\input{common/conman_post}
\input{common/kargs_post}

\subsubsection{Optionally configure stateful provisioning}
\input{common/stateful}

\vspace*{0.35cm}
\subsection{Boot compute nodes} \label{sec:boot_computes}
\input{common/reset_computes} 

%\vspace*{-0.50cm}
\section{Install \OHPC{} Development Components}
\input{common/dev_intro.tex}

%\vspace*{-0.15cm}
\subsection{Development Tools} \label{sec:install_dev_tools}
\input{common/dev_tools}

%\vspace*{-0.15cm}
\subsection{Compilers} \label{sec:install_compilers}
\input{common/compilers}

%\clearpage
\subsection{MPI Stacks} \label{sec:mpi}
For MPI development and runtime support, \OHPC{} provides pre-packaged builds
for a variety of MPI families and transport layers. Currently available options
and their applicability to various network transports are summarized in
Table~\ref{table:mpi}.  The command that follows installs a starting set of MPI
families that are compatible with both ethernet and high-speed fabrics. 

\iftoggleverb{isx86}
% x86_64

\begin{table}[h]
\caption{Available MPI variants} \label{table:mpi}
\centering
\begin{tabular}{@{\hspace*{0.2cm}} *5l @{}}    \toprule
                                  & Ethernet (TCP)                 & \InfiniBand{}                  & \IntelR{} Omni-Path            \\ \midrule
\rowcolor{black!10} MPICH (ofi) & \multicolumn{1}{c}{\checkmark} & \multicolumn{1}{c}{\checkmark} & \multicolumn{1}{c}{\checkmark} \\
 MPICH (ucx)       & \multicolumn{1}{c}{\checkmark} & \multicolumn{1}{c}{\checkmark} & \multicolumn{1}{c}{\checkmark} \\
\rowcolor{black!10} MVAPICH2                          &                                & \multicolumn{1}{c}{\checkmark} &                                \\
MVAPICH2 (psm2) &                              &                                & \multicolumn{1}{c}{\checkmark} \\
\rowcolor{black!10} OpenMPI (ofi/ucx)            & \multicolumn{1}{c}{\checkmark} & \multicolumn{1}{c}{\checkmark} & \multicolumn{1}{c}{\checkmark} \\
%\rowcolor{black!10} OpenMPI (PMIx) & \multicolumn{1}{c}{\checkmark} & \multicolumn{1}{c}{\checkmark} & \multicolumn{1}{c}{\checkmark} \\ \bottomrule
\end{tabular}
\end{table}

\else
% aarch64

\begin{table}[h]
\caption{Available MPI builds} \label{table:mpi}
\centering
\begin{tabular}{@{\hspace*{0.2cm}} *5l @{}}    \toprule
                                  & Ethernet (TCP)                 & \InfiniBand{}                              \\ \midrule
\rowcolor{black!10} MPICH         & \multicolumn{1}{c}{\checkmark} &                                            \\
\rowcolor{black!10} OpenMPI                           & \multicolumn{1}{c}{\checkmark} & \multicolumn{1}{c}{\checkmark} \\
\end{tabular}
\end{table}

\fi


% begin_ohpc_run
% ohpc_comment_header Install MPI Stacks \ref{sec:mpi}
% ohpc_command if [[ ${enable_mpi_defaults} -eq 1 ]];then
% ohpc_indent 5
\begin{lstlisting}[language=bash]
[sms](*\#*) (*\install*) openmpi4-gnu12-ohpc mpich-ofi-gnu12-ohpc
\end{lstlisting}
% ohpc_indent 0
% ohpc_command fi
% end_ohpc_run

Note that OpenHPC 2.x introduces the use of two related transport layers for
the MPICH and OpenMPI builds that support a variety of underlying
fabrics: \href{https://www.openucx.org}{UCX} (Unified Communication X)
and \href{https://ofiwg.github.io/libfabric/}{OFI} (OpenFabrics interfaces).
In the case of OpenMPI, a monolithic build is provided which supports both
transports and end-users can customize their runtime preferences with
environment variables. For MPICH, two separate builds are provided and the
example above highlighted installing the {\texttt ofi} variant.  However, the
packaging is designed such that both versions can be installed simultaneously
and users can switch between the two via normal module command
semantics. Alternatively, a site can choose to install the {\texttt ucx} variant
instead as a drop-in MPICH replacement:

\begin{lstlisting}[language=bash]
[sms](*\#*) (*\install*) mpich-ucx-gnu12-ohpc
\end{lstlisting}

In the case where both MPICH variants are installed, two modules will be
visible in the end-user environment and an example of this configuration is
highlighted is below. 

\begin{lstlisting}[language=bash]
[sms](*\#*) module avail mpich

-------------------- /opt/ohpc/pub/moduledeps/gnu12---------------------
   mpich/3.4.3-ofi    mpich/3.4.3-ucx (D)
\end{lstlisting}

If your system includes \InfiniBand{} and you enabled underlying support in
\S\ref{sec:add_ofed} and \S\ref{sec:addl_customizations}, an additional
MVAPICH2 family is available for use:

% begin_ohpc_run
% ohpc_validation_newline
% ohpc_command if [[ ${enable_ib} -eq 1 ]];then
% ohpc_indent 5
\begin{lstlisting}[language=bash]
[sms](*\#*) (*\install*) mvapich2-gnu12-ohpc
\end{lstlisting}
% ohpc_indent 0
% ohpc_command fi
% end_ohpc_run

Alternatively, if your system includes \IntelR{} \OmniPath{}, use the (\texttt{psm2})
variant of MVAPICH2 instead:

% begin_ohpc_run
% ohpc_command if [[ ${enable_opa} -eq 1 ]];then
% ohpc_indent 5
\begin{lstlisting}[language=bash]
[sms](*\#*) (*\install*) mvapich2-psm2-gnu12-ohpc
\end{lstlisting}
% ohpc_indent 0
% ohpc_command fi
% end_ohpc_run

%%--
%% https://github.com/openhpc/ohpc/issues/1273
%% disabling until we can get pmix/openmpi/slurm to play nicely
%%--
%% An additional OpenMPI build variant is listed in Table~\ref{table:mpi} which
%% enables \href{https://pmix.github.io/pmix/}{\color{blue}{PMIx}} job launch
%% support for use with \SLURM{}. This optional variant is
%% available as \texttt{openmpi4-pmix-slurm-gnu9-ohpc}.


\subsection{Performance Tools} \label{sec:install_perf_tools}
\input{common/perf_tools}

\subsection{Setup default development environment}
\input{common/default_dev}

\subsection{3rd Party Libraries and Tools} \label{sec:3rdparty}
\input{common/third_party_libs_intro}
\input{common/third_party_libs_petsc_centos}
\input{common/third_party_libs}
\input{common/third_party_mpi_libs_x86}

%\vspace*{0.4cm}
\subsection{Optional Development Tool Builds} \label{sec:3rdparty_intel}
%\vspace*{0.2cm}
\input{common/oneapi_enabled_builds}

%\clearpage
\section{Resource Manager Startup} \label{sec:rms_startup}
\input{common/openpbs_startup}

\section{Run a Test Job} \label{sec:test_job}
\input{common/openpbs_test_job}

\clearpage
\appendix
%\section*{Appendices}
{\bf \LARGE \centerline{Appendices}} \vspace*{0.2cm}

\addcontentsline{toc}{section}{Appendices}
\renewcommand{\thesubsection}{\Alph{subsection}}

\input{common/automation_appendix}
\input{common/upgrade}
\input{common/test_suite}
\input{common/customization_appendix_centos}
\input{manifest}
\input{common/signature}


\end{document}

