\documentclass[letterpaper]{article}
\usepackage{common/ohpc-doc}
\setcounter{secnumdepth}{5}
\setcounter{tocdepth}{5}

% Include git variables
\input{vc.tex}

% Define Base OS and other local macros
\newcommand{\baseOS}{Rocky 8.8}
\newcommand{\OSRepo}{Rocky\_8.8}
\newcommand{\OSTree}{EL\_8}
\newcommand{\OSTag}{el8}
\newcommand{\baseos}{rocky8.8}
\newcommand{\baseosshort}{rocky8}
\newcommand{\provisioner}{Warewulf}
\newcommand{\provheader}{\provisioner{}}
\newcommand{\rms}{SLURM}
\newcommand{\rmsshort}{slurm}
\newcommand{\arch}{aarch64}

% Define package manager commands
\newcommand{\pkgmgr}{yum}
\newcommand{\addrepo}{wget -P /etc/yum.repos.d}
\newcommand{\chrootaddrepo}{wget -P \$CHROOT/etc/yum.repos.d}
\newcommand{\clean}{yum clean expire-cache}
\newcommand{\chrootclean}{yum --installroot=\$CHROOT clean expire-cache}
\newcommand{\install}{yum -y install}
\newcommand{\chrootinstall}{yum -y --installroot=\$CHROOT install}
\newcommand{\groupinstall}{yum -y groupinstall}
\newcommand{\groupchrootinstall}{yum -y --installroot=\$CHROOT groupinstall}
\newcommand{\remove}{yum -y remove}
\newcommand{\upgrade}{yum -y upgrade}
\newcommand{\chrootupgrade}{yum -y --installroot=\$CHROOT upgrade}
\newcommand{\tftppkg}{syslinux-tftpboot}

% boolean for os-specific formatting
\toggletrue{isaarch}
\toggletrue{isCentOS}
\toggletrue{isCentOS_ww_slurm_aarch}
\toggletrue{isSLURM}
\toggletrue{isWarewulf}

\begin{document}
\graphicspath{{common/figures/}}
\thispagestyle{empty}

% Title Page --------------------------------------------------------
\input{common/title}
% Disclaimer
\input{common/legal} 

\newpage
\tableofcontents
\newpage

% Introduction  ----------------------------------------------------


\section{Introduction} \label{sec:introduction}
\input{common/install_header}
\input{common/intro} \\

\input{common/base_edition/edition}
\input{common/audience}
\input{common/requirements}
\input{common/inputs}

% begin_ohpc_run
% ohpc_validation_newline
% ohpc_validation_comment Verify OpenHPC repository has been enabled before proceeding
% ohpc_validation_newline
% ohpc_command yum repolist | grep -q OpenHPC
% ohpc_command if [ $? -ne 0 ];then
% ohpc_command    echo "Error: OpenHPC repository must be enabled locally"
% ohpc_command    exit 1
% ohpc_command fi
% end_ohpc_run

% Base Operating System --------------------------------------------

\section{Install Base Operating System (BOS)}
\input{common/bos}

% begin_ohpc_run
% ohpc_validation_newline
% ohpc_validation_comment Disable firewall 
\begin{lstlisting}[language=bash,keywords={}]
[sms](*\#*) systemctl disable firewalld
[sms](*\#*) systemctl stop firewalld
\end{lstlisting}
% end_ohpc_run

% ------------------------------------------------------------------

\section{Install \OHPC{} Components} \label{sec:basic_install}
\input{common/install_ohpc_components_intro.tex}

\subsection{Enable \OHPC{} repository for local use} \label{sec:enable_repo}
\input{common/enable_ohpc_repo}
\input{common/rocky_repos}
\input{common/automation}

\subsection{Add provisioning services on {\em master} node} \label{sec:add_provisioning}
\input{common/install_provisioning_intro}
\input{common/enable_pxe}
\input{common/time}

\subsection{Add resource management services on {\em master} node} \label{sec:add_rm}
\OHPC{} provides multiple options for distributed resource management. 
The following command adds the \SLURM{} workload manager server components to the
chosen {\em master} host. Note that client-side components will be added to
the corresponding compute image in a subsequent step.

% begin_ohpc_run
% ohpc_comment_header Add resource management services on master node \ref{sec:add_rm}
\begin{lstlisting}[language=bash,keywords={}]
# Install slurm server meta-package
[sms](*\#*) (*\install*) ohpc-slurm-server

# Use ohpc-provided file for starting SLURM configuration
[sms](*\#*) cp /etc/slurm/slurm.conf.ohpc /etc/slurm/slurm.conf
# Setup default cgroups file
[sms](*\#*) cp /etc/slurm/cgroup.conf.example /etc/slurm/cgroup.conf

# Identify resource manager hostname on master host
[sms](*\#*) perl -pi -e "s/SlurmctldHost=\S+/SlurmctldHost=${sms_name}/" /etc/slurm/slurm.conf
\end{lstlisting}
% end_ohpc_run

There are a wide variety of configuration options and plugins available
for \SLURM{} and the example config file illustrated above targets a fairly
basic installation. In particular, job completion data will be stored in a text
file (\texttt{/var/log/slurm\_jobcomp.log)} that can be used to log simple
accounting information. Sites who desire more detailed information, or want to
aggregate accounting data from multiple clusters, will likely want to enable the
database accounting back-end.  This requires a number of additional local modifications
(on top of installing \texttt{slurm-slurmdbd-ohpc}), and users are advised to
consult the online \href{https://slurm.schedmd.com/accounting.html}{\color{blue}{documentation}}
for more detailed information on setting up a database configuration for \SLURM{}.

\begin{center}
\begin{tcolorbox}[]
  \small SLURM requires enumeration of the physical hardware characteristics for
  compute nodes under its control. In particular, three configuration parameters
  combine to define consumable compute resources: {\em Sockets}, {\em
  CoresPerSocket}, and {\em ThreadsPerCore}. The default configuration file
  provided via \OHPC{} assumes the nodes are named c1-c4 and are dual-socket, 8
  cores per socket, and two threads per core for this 4-node example. If this
  does not reflect your local hardware, please update the configuration file at
  \path{/etc/slurm/slurm.conf} accordingly to match your nodes names and
  particular hardware. Be sure to run \texttt{scontrol reconfigure} to notify
  SLURM of the changes. Note that the SLURM project provides an easy-to-use
  online configuration tool that can be accessed
  \href{https://slurm.schedmd.com/configurator.html}{\color{blue} here}.
\end{tcolorbox}
\end{center}

% begin_ohpc_run
% ohpc_comment_header Update node configuration for slurm.conf
% ohpc_command if [[ ${update_slurm_nodeconfig} -eq 1 ]];then
% ohpc_indent 5
% ohpc_command perl -pi -e "s/^NodeName=.+$/#/" /etc/slurm/slurm.conf
% ohpc_command perl -pi -e "s/ Nodes=c\S+ / Nodes=c[1-$num_computes] /" /etc/slurm/slurm.conf
% ohpc_command echo -e ${slurm_node_config} >> /etc/slurm/slurm.conf
% ohpc_indent 0
% ohpc_command fi
% end_ohpc_run

Other versions of this guide are available that describe installation of alternate
resource management systems, and they can be found in the \texttt{docs-ohpc}
package.



%% Add if/when IB is available for testing
%% \subsection{Optionally add \InfiniBand{} support services on {\em master} node} \label{sec:add_ofed}
%% \input{common/ibsupport_sms_centos}

%\vspace*{-0.15cm}
\subsection{Complete basic Warewulf setup for {\em master} node} \label{sec:setup_ww}
\input{common/warewulf_setup}
\input{common/warewulf_setup_centos}

\subsection{Define {\em compute} image for provisioning}
\input{common/warewulf_mkchroot_rocky}

\subsubsection{Add \OHPC{} components} \label{sec:add_components}
\input{common/add_to_compute_chroot_intro}

% begin_ohpc_run
% ohpc_validation_comment Add OpenHPC components to compute instance
\begin{lstlisting}[language=bash,literate={-}{-}1,keywords={},upquote=true]
# copy credential files into $CHROOT to ensure consistent uid/gids for slurm/munge at
# install. Note that these will be synchronized with future updates via the
# provisioning system.
[sms](*\#*) cp /etc/passwd /etc/group  $CHROOT/etc

# Add Slurm client support meta-package and enable munge and slurmd
[sms](*\#*) (*\chrootinstall*) ohpc-slurm-client
[sms](*\#*) chroot $CHROOT systemctl enable munge
[sms](*\#*) chroot $CHROOT systemctl enable slurmd

# Register Slurm server with computes (using "configless" option)
[sms](*\#*) echo SLURMD_OPTIONS="--conf-server ${sms_ip}" > $CHROOT/etc/sysconfig/slurmd

# Add Network Time Protocol (NTP) support
[sms](*\#*) (*\chrootinstall*) chrony
# Identify master host as local NTP server
[sms](*\#*) echo "server ${sms_ip} iburst" >> $CHROOT/etc/chrony.conf

# Add kernel drivers (matching kernel version on SMS node)
[sms](*\#*) (*\chrootinstall*) kernel-`uname -r`

# Include modules user environment
[sms](*\#*) (*\chrootinstall*) lmod-ohpc
\end{lstlisting}
% end_ohpc_run

\subsubsection{Customize system configuration} \label{sec:master_customization}
%\input{common/warewulf_chroot_customize_centos}
\input{common/warewulf_chroot_customize_rhel_aarch}
\input{common/restart_nfs}

% Additional commands when additional computes are requested

% begin_ohpc_run
% ohpc_validation_newline
% ohpc_validation_comment Update basic slurm configuration if additional computes defined
% ohpc_command if [ ${num_computes} -gt 4 ];then
% ohpc_command    perl -pi -e "s/^NodeName=(\S+)/NodeName=${compute_prefix}[1-${num_computes}]/" /etc/slurm/slurm.conf
% ohpc_command    perl -pi -e "s/^PartitionName=normal Nodes=(\S+)/PartitionName=normal Nodes=${compute_prefix}[1-${num_computes}]/" /etc/slurm/slurm.conf

% ohpc_command    perl -pi -e "s/^NodeName=(\S+)/NodeName=${compute_prefix}[1-${num_computes}]/" $CHROOT/etc/slurm/slurm.conf
% ohpc_command    perl -pi -e "s/^PartitionName=normal Nodes=(\S+)/PartitionName=normal Nodes=${compute_prefix}[1-${num_computes}]/" $CHROOT/etc/slurm/slurm.conf
% ohpc_command fi
% end_ohpc_run

%\clearpage
\subsubsection{Additional Customization ({\em optional})} \label{sec:addl_customizations}
\input{common/compute_customizations_intro}

%% Add if/when IB is available for testing
%% \paragraph{Increase locked memory limits}
%% \input{common/memlimits}

\paragraph{Enable ssh control via resource manager} 
\input{common/slurm_pam}

%%\paragraph{Add \Lustre{} client} \label{sec:lustre_client}
%%\input{common/lustre-client}
%%\input{common/lustre-client-centos}
%%\vspace*{0.5cm}
%%\input{common/lustre-client-post}

%\clearpage
\paragraph{Enable forwarding of system logs} \label{sec:add_syslog}
\input{common/syslog}

\paragraph{Add \Nagios{} monitoring} \label{sec:add_nagios}
\input{common/nagios}

\paragraph{Add \clustershell{}}
\input{common/clustershell}

\paragraph{Add \genders{}}
\input{common/genders}

\paragraph{Add Magpie}
\input{common/magpie}

\paragraph{Add \conman{}} \label{sec:add_conman}
\input{common/conman}

\paragraph{Add \nhc{}} \label{sec:add_nhc}
\input{common/nhc}
\input{common/nhc_slurm}

\vspace*{-0.25cm}
\subsubsection{Import files} \label{sec:file_import}
\input{common/import_ww_files}

\input{common/import_ww_files_slurm}
%% \input{common/import_ww_files_ib_centos}
\input{common/finalize_provisioning}
\input{common/add_ww_hosts_intro}
\input{common/add_ww_hosts_slurm}
\input{common/add_ww_hosts_finalize}

\vspace*{0.2cm}
\subsubsection{Optional kernel arguments} \label{sec:optional_kargs}
\input{common/charliecloud_centos_warewulf_post}
\input{common/conman_post}
\input{common/kargs_post}

\subsubsection{Optionally configure stateful provisioning}
\input{common/stateful}

\vspace*{-0.1cm}
\subsection{Boot compute nodes} \label{sec:boot_computes}
\input{common/reset_computes} 

%\vspace*{-0.50cm}
\section{Install \OHPC{} Development Components} \label{sec:install_dev}
\input{common/dev_intro.tex}

%\vspace*{-0.15cm}
\subsection{Development Tools} \label{sec:install_dev_tools}
\input{common/dev_tools}

%\vspace*{-0.15cm}
\subsection{Compilers} \label{sec:install_compilers}
\input{common/compilers}

%\clearpage
\vspace*{0.25cm}
\subsection{MPI Stacks} \label{sec:mpi}
For MPI development and runtime support, \OHPC{} provides pre-packaged builds
for two MPI families that are compatible with both ethernet and high-speed
fabrics.  These MPI stacks can be installed as follows:

% begin_ohpc_run
% ohpc_comment_header Install MPI Stacks \ref{sec:mpi}
% ohpc_command if [[ ${enable_mpi_defaults} -eq 1 ]];then
% ohpc_indent 5
\begin{lstlisting}[language=bash]
[sms](*\#*) (*\install*) openmpi4-pmix-gnu12-ohpc mpich-ofi-gnu12-ohpc
\end{lstlisting}
% ohpc_indent 0
% ohpc_command fi
% end_ohpc_run

Note that OpenHPC 2.x introduces the use of two related transport layers for
the MPICH and OpenMPI builds that support a variety of underlying
fabrics: \href{https://www.openucx.org}{UCX} (Unified Communication X)
and \href{https://ofiwg.github.io/libfabric/}{OFI} (OpenFabrics interfaces).
In the case of OpenMPI, a monolithic build is provided which supports both
transports and end-users can customize their runtime preferences with
environment variables. For MPICH, two separate builds are provided and the
example above highlighted installing the {\texttt ofi} variant.  However, the
packaging is designed such that both versions can be installed simultaneously
and users can switch between the two via normal module command
semantics. Alternatively, a site can choose to install the {\texttt ucx} variant
instead as a drop-in MPICH replacement:

% begin_ohpc_run
% ohpc_command if [[ ${enable_mpich_ucx} -eq 1 ]];then
% ohpc_indent 5
\begin{lstlisting}[language=bash]
[sms](*\#*) (*\install*) mpich-ucx-gnu12-ohpc
\end{lstlisting}
% ohpc_indent 0
% ohpc_command fi
% end_ohpc_run

In the case where both MPICH variants are installed, two modules will be
visible in the end-user environment and an example of this configuration is
highlighted is below.

\begin{lstlisting}[language=bash]
[sms](*\#*) module avail mpich

-------------------- /opt/ohpc/pub/moduledeps/gnu12 ---------------------
   mpich/3.4.3-ofi    mpich/3.4.3-ucx (D)
\end{lstlisting}







\subsection{Performance Tools} \label{sec:install_perf_tools}
\input{common/perf_tools}

\subsection{Setup default development environment}
\input{common/default_dev}

%\clearpage
\subsection{3rd Party Libraries and Tools} \label{sec:3rdparty}
\input{common/third_party_libs_intro}
\input{common/third_party_libs_petsc_centos}
\input{common/third_party_libs}
\input{common/third_party_mpi_libs_aarch}

\subsection{Optional Development Tool Builds} \label{sec:3rdparty_arm}
\input{common/armhpc_enabled_builds}

\clearpage
\section{Resource Manager Startup} \label{sec:rms_startup}
\input{common/slurm_startup}

\section{Run a Test Job} \label{sec:test_job}
\input{common/slurm_test_job}

\clearpage
\appendix
{\bf \LARGE \centerline{Appendices}} \vspace*{0.2cm}

\addcontentsline{toc}{section}{Appendices}
\renewcommand{\thesubsection}{\Alph{subsection}}

\input{common/automation_appendix}
\input{common/upgrade}
\input{common/test_suite}
\input{common/customization_appendix_centos}
\input{manifest}
\input{common/signature}


\end{document}

