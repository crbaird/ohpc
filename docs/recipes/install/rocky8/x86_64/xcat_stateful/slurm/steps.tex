\documentclass[letterpaper]{article}
\usepackage{common/ohpc-doc}
\setcounter{secnumdepth}{5}
\setcounter{tocdepth}{5}

% Include git variables
\input{vc.tex}

% Define Base OS and other local macros
\newcommand{\baseOS}{CentOS8.4}
\newcommand{\OSRepo}{CentOS\_8.4}
\newcommand{\OSTree}{CentOS\_8}
\newcommand{\OSTag}{el8}
\newcommand{\baseos}{centos8.4}
\newcommand{\baseosshort}{centos8}
\newcommand{\provisioner}{xCAT}
\newcommand{\provheader}{xCAT (stateful)}
\newcommand{\rms}{SLURM}
\newcommand{\rmsshort}{slurm}
\newcommand{\arch}{x86\_64}
\newcommand{\installimage}{install}
%%% WARNING: Hack below. The version should be read from ohpc-doc.sty, but the
%%% perl parsing script does not read that file. This works for one release, but
%%% needs a proper fix.
\newcommand{\VERLONG}{2.0}

% Define package manager commands
\newcommand{\pkgmgr}{yum}
\newcommand{\addrepo}{wget -P /etc/yum.repos.d}
\newcommand{\chrootaddrepo}{wget -P \$CHROOT/etc/yum.repos.d}
\newcommand{\clean}{yum clean expire-cache}
\newcommand{\chrootclean}{yum --installroot=\$CHROOT clean expire-cache}
\newcommand{\install}{yum -y install}
\newcommand{\chrootinstall}{psh compute yum -y install}
\newcommand{\groupinstall}{yum -y groupinstall}
\newcommand{\groupchrootinstall}{psh compute yum -y groupinstall}
\newcommand{\remove}{yum -y remove}
\newcommand{\upgrade}{yum -y upgrade}
\newcommand{\chrootupgrade}{yum -y --installroot=\$CHROOT upgrade}
\newcommand{\tftppkg}{syslinux-tftpboot}
\newcommand{\beegfsrepo}{https://www.beegfs.io/release/beegfs\_7.2.1/dists/beegfs-rhel8.repo}

% boolean for os-specific formatting
\toggletrue{isCentOS}
\toggletrue{isCentOS_ww_slurm_x86}
\toggletrue{isSLURM}
\toggletrue{isx86}
\toggletrue{isxCAT}
\toggletrue{isxCATstateful}
\toggletrue{isCentOS_x86}

\begin{document}
\graphicspath{{common/figures/}}
\thispagestyle{empty}

% Title Page
\input{common/title}
% Disclaimer 
\input{common/legal} 

\newpage
\tableofcontents
\newpage

% Introduction  --------------------------------------------------

\section{Introduction} \label{sec:introduction}
\input{common/install_header}
\input{common/intro} \\

\input{common/base_edition/edition}
\input{common/audience}
\input{common/requirements}
\input{common/inputs}


% Base Operating System --------------------------------------------

\vspace*{0.2cm}
\section{Install Base Operating System (BOS)}
\input{common/bos}

%\clearpage 
% begin_ohpc_run
% ohpc_validation_newline
% ohpc_validation_comment Disable firewall 
\begin{lstlisting}[language=bash,keywords={}]
[sms](*\#*) systemctl disable firewalld
[sms](*\#*) systemctl stop firewalld
\end{lstlisting}
% end_ohpc_run

% ------------------------------------------------------------------

\section{Install \xCAT{} and Provision Nodes with BOS} \label{sec:provision_compute_bos}
\input{common/xcat_stateful_compute_bos_intro}

\subsection{Enable \xCAT{} repository for local use} \label{sec:enable_xcat}
\input{common/enable_xcat_repo}

\noindent \xCAT{} has a number of dependencies that are required for
installation that are housed in separate public repositories for various
distributions. To enable for local use, issue the following:

% begin_ohpc_run
\begin{lstlisting}[language=bash,keywords={},basicstyle=\fontencoding{T1}\fontsize{8.0}{10}\ttfamily,literate={ARCH}{\arch{}}1 {-}{-}1]
[sms](*\#*) (*\install*) centos-release-stream
[sms](*\#*) (*\addrepo*) https://xcat.org/files/xcat/repos/yum/xcat-dep/rh8/ARCH/xcat-dep.repo
\end{lstlisting}
% end_ohpc_run

\subsection{Add provisioning services on {\em master} node} \label{sec:add_provisioning}
\input{common/install_provisioning_xcat_intro_stateful}
%\input{common/enable_pxe}

\vspace*{-0.15cm}
\subsection{Complete basic \xCAT{} setup for {\em master} node} \label{sec:setup_xcat}
\input{common/xcat_setup}


\subsection{Define {\em compute} image for provisioning}
\input{common/xcat_init_os_images_centos}

\clearpage
\subsection{Add compute nodes into \xCAT{} database} \label{sec:xcat_add_nodes}
\input{common/add_xcat_hosts_intro}

%\vspace*{-0.25cm}
\subsection{Boot compute nodes} \label{sec:boot_computes}
\input{common/reset_computes_xcat} 




\section{Install \OHPC{} Components} \label{sec:basic_install}
\input{common/install_ohpc_components_intro}


\subsection{Enable \OHPC{} repository for local use} \label{sec:enable_repo}
\input{common/enable_local_ohpc_repo}

% begin_ohpc_run
% ohpc_validation_newline
% ohpc_validation_comment Verify OpenHPC repository has been enabled before proceeding
% ohpc_validation_newline
% ohpc_command yum repolist | grep -q OpenHPC
% ohpc_command if [ $? -ne 0 ];then
% ohpc_command    echo "Error: OpenHPC repository must be enabled locally"
% ohpc_command    exit 1
% ohpc_command fi
% end_ohpc_run


In addition to the \OHPC{} and \xCAT{} package repositories, the {\em master} host also
requires access to the standard base OS distro repositories in order to resolve
necessary dependencies. For \baseOS{}, the requirements are to have access to
both the base OS and EPEL repositories for which mirrors are freely available online:

\begin{itemize*}
\item CentOS-8 - Base 8.3.2011
  (e.g. \href{http://mirror.centos.org/centos-8/8/BaseOS/x86\_64/os}
             {\color{blue}{http://mirror.centos.org/centos-8/8/BaseOS/x86\_64/os}} )
\item EPEL 8 (e.g. \href{http://download.fedoraproject.org/pub/epel/8/Everything/x86\_64}
                        {\color{blue}{http://download.fedoraproject.org/pub/epel/8/Everything/x86\_64}} )
\end{itemize*}

\noindent The public EPEL repository is enabled by installing
\texttt{epel-release} package. Note that this requires the CentOS Extras
repository, which is shipped with CentOS and is enabled by default.

% begin_ohpc_run
\begin{lstlisting}[language=bash,keywords={},basicstyle=\fontencoding{T1}\fontsize{8.0}{10}\ttfamily,literate={ARCH}{\arch{}}1 {-}{-}1]
[sms](*\#*)  (*\install*) epel-release
\end{lstlisting}
% end_ohpc_run

Now \OHPC{} packages can be installed. To add the base package on the SMS
issue the following
% begin_ohpc_run
\begin{lstlisting}[language=bash,keywords={},basicstyle=\fontencoding{T1}\fontsize{8.0}{10}\ttfamily,literate={ARCH}{\arch{}}1 {-}{-}1]
[sms](*\#*)  (*\install*) ohpc-base
\end{lstlisting}
% end_ohpc_run


\input{common/automation}


\subsection{Setup time synchronization service on {\em master} node} \label{sec:add_ntp}
\input{common/time}

\subsection{Add resource management services on {\em master} node} \label{sec:add_rm}
\OHPC{} provides multiple options for distributed resource management. 
The following command adds the \SLURM{} workload manager server components to the
chosen {\em master} host. Note that client-side components will be added to
the corresponding compute image in a subsequent step.

% begin_ohpc_run
% ohpc_comment_header Add resource management services on master node \ref{sec:add_rm}
\begin{lstlisting}[language=bash,keywords={}]
# Install slurm server meta-package
[sms](*\#*) (*\install*) ohpc-slurm-server

# Use ohpc-provided file for starting SLURM configuration
[sms](*\#*) cp /etc/slurm/slurm.conf.ohpc /etc/slurm/slurm.conf
# Setup default cgroups file
[sms](*\#*) cp /etc/slurm/cgroup.conf.example /etc/slurm/cgroup.conf

# Identify resource manager hostname on master host
[sms](*\#*) perl -pi -e "s/SlurmctldHost=\S+/SlurmctldHost=${sms_name}/" /etc/slurm/slurm.conf
\end{lstlisting}
% end_ohpc_run

There are a wide variety of configuration options and plugins available
for \SLURM{} and the example config file illustrated above targets a fairly
basic installation. In particular, job completion data will be stored in a text
file (\texttt{/var/log/slurm\_jobcomp.log)} that can be used to log simple
accounting information. Sites who desire more detailed information, or want to
aggregate accounting data from multiple clusters, will likely want to enable the
database accounting back-end.  This requires a number of additional local modifications
(on top of installing \texttt{slurm-slurmdbd-ohpc}), and users are advised to
consult the online \href{https://slurm.schedmd.com/accounting.html}{\color{blue}{documentation}}
for more detailed information on setting up a database configuration for \SLURM{}.

\begin{center}
\begin{tcolorbox}[]
  \small SLURM requires enumeration of the physical hardware characteristics for
  compute nodes under its control. In particular, three configuration parameters
  combine to define consumable compute resources: {\em Sockets}, {\em
  CoresPerSocket}, and {\em ThreadsPerCore}. The default configuration file
  provided via \OHPC{} assumes the nodes are named c1-c4 and are dual-socket, 8
  cores per socket, and two threads per core for this 4-node example. If this
  does not reflect your local hardware, please update the configuration file at
  \path{/etc/slurm/slurm.conf} accordingly to match your nodes names and
  particular hardware. Be sure to run \texttt{scontrol reconfigure} to notify
  SLURM of the changes. Note that the SLURM project provides an easy-to-use
  online configuration tool that can be accessed
  \href{https://slurm.schedmd.com/configurator.html}{\color{blue} here}.
\end{tcolorbox}
\end{center}

% begin_ohpc_run
% ohpc_comment_header Update node configuration for slurm.conf
% ohpc_command if [[ ${update_slurm_nodeconfig} -eq 1 ]];then
% ohpc_indent 5
% ohpc_command perl -pi -e "s/^NodeName=.+$/#/" /etc/slurm/slurm.conf
% ohpc_command perl -pi -e "s/ Nodes=c\S+ / Nodes=c[1-$num_computes] /" /etc/slurm/slurm.conf
% ohpc_command echo -e ${slurm_node_config} >> /etc/slurm/slurm.conf
% ohpc_indent 0
% ohpc_command fi
% end_ohpc_run

Other versions of this guide are available that describe installation of alternate
resource management systems, and they can be found in the \texttt{docs-ohpc}
package.



\subsection{Optionally add \InfiniBand{} support services on {\em master} node} \label{sec:add_ofed}
\input{common/ibsupport_sms_centos}
\vspace*{0.3cm}
\subsection{Optionally add \OmniPath{} support services on {\em master} node} \label{sec:add_opa}
\input{common/opasupport_sms_centos}

%\vspace*{0.5cm}
\clearpage
\subsubsection{Add \OHPC{} components to {\em compute} nodes} \label{sec:add_components}
\input{common/add_to_compute_stateful_xcat_intro}

%\newpage
% begin_ohpc_run
% ohpc_validation_comment Add OpenHPC components to compute instance
\begin{lstlisting}[language=bash,literate={-}{-}1,keywords={},upquote=true]
# Add Slurm client support meta-package
[sms](*\#*) (*\chrootinstall*) ohpc-slurm-client

# Add Network Time Protocol (NTP) support
[sms](*\#*) (*\chrootinstall*) ntp

# Add kernel drivers
[sms](*\#*) (*\chrootinstall*) kernel

# Include modules user environment
[sms](*\#*) (*\chrootinstall*)  --enablerepo=powertools lmod-ohpc
\end{lstlisting}
% end_ohpc_run

% ohpc_comment_header Optionally add InfiniBand support services in compute node image \ref{sec:add_components}
% ohpc_command if [[ ${enable_ib} -eq 1 ]];then
% ohpc_indent 5
\begin{lstlisting}[language=bash,literate={-}{-}1,keywords={},upquote=true]
# Optionally add IB support and enable
[sms](*\#*) (*\groupchrootinstall*) "InfiniBand Support"
\end{lstlisting}
% ohpc_indent 0
% ohpc_command fi
% end_ohpc_run

\vspace*{-0.25cm}
\subsubsection{Customize system configuration} \label{sec:master_customization}
\input{common/xcat_stateful_customize_centos}

% Additional commands when additional computes are requested

% begin_ohpc_run
% ohpc_validation_newline
% ohpc_validation_comment Update basic slurm configuration if additional computes defined
% ohpc_validation_comment This is performed on the SMS, nodes will pick it up config file is copied there later
% ohpc_command if [ ${num_computes} -gt 4 ];then
% ohpc_command    perl -pi -e "s/^NodeName=(\S+)/NodeName=${compute_prefix}[1-${num_computes}]/" /etc/slurm/slurm.conf
% ohpc_command    perl -pi -e "s/^PartitionName=normal Nodes=(\S+)/PartitionName=normal Nodes=${compute_prefix}[1-${num_computes}]/" /etc/slurm/slurm.conf
% ohpc_command fi
% end_ohpc_run

%\clearpage
\subsubsection{Additional Customization ({\em optional})} \label{sec:addl_customizations}
\input{common/compute_customizations_intro}

\paragraph{Increase locked memory limits}
\input{common/memlimits_stateful}

\paragraph{Enable ssh control via resource manager} 
\input{common/slurm_pam_stateful}

\paragraph{Add \Lustre{} client} \label{sec:lustre_client}
\input{common/lustre-client}
\input{common/lustre-client-centos-stateful}
\input{common/lustre-client-post-stateful}

\paragraph{Add \Nagios{} monitoring} \label{sec:add_nagios}
\input{common/nagios_stateful}

\vspace*{0.4cm}

\paragraph{Add \clustershell{}}
\input{common/clustershell}

\paragraph{Add \genders{}}
\input{common/genders}

\paragraph{Add Magpie}
\input{common/magpie}

\paragraph{Add \conman{}} \label{sec:add_conman}
\input{common/conman}

\paragraph{Add \nhc{}} \label{sec:add_nhc}
\input{common/nhc}
\input{common/nhc_slurm}

%\subsubsection{Identify files for synchronization} \label{sec:file_import}
%\input{common/import_xcat_files}
%\input{common/import_xcat_files_slurm}

%%%\subsubsection{Optional kernel arguments} \label{sec:optional_kargs}
%%%\input{common/conman_post}

\section{Install \OHPC{} Development Components}
\input{common/dev_intro.tex}

%\vspace*{-0.15cm}
%\clearpage
\subsection{Development Tools} \label{sec:install_dev_tools}
\input{common/dev_tools}

\vspace*{-0.15cm}
\subsection{Compilers} \label{sec:install_compilers}
\input{common/compilers}

%\clearpage
\subsection{MPI Stacks} \label{sec:mpi}
For MPI development and runtime support, \OHPC{} provides pre-packaged builds
for a variety of MPI families and transport layers. Currently available options
and their applicability to various network transports are summarized in
Table~\ref{table:mpi}.  The command that follows installs a starting set of MPI
families that are compatible with both ethernet and high-speed fabrics.

\iftoggleverb{isx86}
% x86_64

\begin{table}[h]
\caption{Available MPI variants} \label{table:mpi}
\centering
\begin{tabular}{@{\hspace*{0.2cm}} *5l @{}}    \toprule
                                  & Ethernet (TCP)                 & \InfiniBand{}                  & \IntelR{} Omni-Path            \\ \midrule
\rowcolor{black!10} MPICH (ofi) & \multicolumn{1}{c}{\checkmark} & \multicolumn{1}{c}{\checkmark} & \multicolumn{1}{c}{\checkmark} \\
 MPICH (ucx)       & \multicolumn{1}{c}{\checkmark} & \multicolumn{1}{c}{\checkmark} & \multicolumn{1}{c}{\checkmark} \\
\rowcolor{black!10} MVAPICH2                          &                                & \multicolumn{1}{c}{\checkmark} &                                \\
MVAPICH2 (psm2) &                              &                                & \multicolumn{1}{c}{\checkmark} \\
\rowcolor{black!10} OpenMPI (ofi/ucx)            & \multicolumn{1}{c}{\checkmark} & \multicolumn{1}{c}{\checkmark} & \multicolumn{1}{c}{\checkmark} \\
%\rowcolor{black!10} OpenMPI (PMIx) & \multicolumn{1}{c}{\checkmark} & \multicolumn{1}{c}{\checkmark} & \multicolumn{1}{c}{\checkmark} \\ \bottomrule
\end{tabular}
\end{table}

\else
% aarch64

\begin{table}[h]
\caption{Available MPI builds} \label{table:mpi}
\centering
\begin{tabular}{@{\hspace*{0.2cm}} *5l @{}}    \toprule
                                  & Ethernet (TCP)                 & \InfiniBand{}                              \\ \midrule
\rowcolor{black!10} MPICH         & \multicolumn{1}{c}{\checkmark} &                                            \\
\rowcolor{black!10} OpenMPI                           & \multicolumn{1}{c}{\checkmark} & \multicolumn{1}{c}{\checkmark} \\
\end{tabular}
\end{table}

\fi

% begin_ohpc_run
% ohpc_comment_header Install MPI Stacks \ref{sec:mpi}
% ohpc_command if [[ ${enable_mpi_defaults} -eq 1 ]];then
% ohpc_indent 5
\begin{lstlisting}[language=bash]
[sms](*\#*) (*\install*) openmpi4-pmix-gnu12-ohpc mpich-ofi-gnu12-ohpc
\end{lstlisting}
% ohpc_indent 0
% ohpc_command fi
% end_ohpc_run

Note that OpenHPC 2.x introduces the use of two related transport layers for
the MPICH and OpenMPI builds that support a variety of underlying
fabrics: \href{https://www.openucx.org}{UCX} (Unified Communication X)
and \href{https://ofiwg.github.io/libfabric/}{OFI} (OpenFabrics interfaces).
In the case of OpenMPI, a monolithic build is provided which supports both
transports and end-users can customize their runtime preferences with
environment variables. For MPICH, two separate builds are provided and the
example above highlighted installing the {\texttt ofi} variant.  However, the
packaging is designed such that both versions can be installed simultaneously
and users can switch between the two via normal module command
semantics. Alternatively, a site can choose to install the {\texttt ucx} variant
instead as a drop-in MPICH replacement:

\begin{lstlisting}[language=bash]
[sms](*\#*) (*\install*) mpich-ucx-gnu12-ohpc
\end{lstlisting}

In the case where both MPICH variants are installed, two modules will be
visible in the end-user environment and an example of this configuration is
highlighted is below.

\begin{lstlisting}[language=bash]
[sms](*\#*) module avail mpich

-------------------- /opt/ohpc/pub/moduledeps/gnu12---------------------
   mpich/3.4.3-ofi    mpich/3.4.3-ucx (D)
\end{lstlisting}

If your system includes \InfiniBand{} and you enabled underlying support in
\S\ref{sec:add_ofed} and \S\ref{sec:addl_customizations}, an additional
MVAPICH2 family is available for use:

% begin_ohpc_run
% ohpc_validation_newline
% ohpc_command if [[ ${enable_ib} -eq 1 ]];then
% ohpc_indent 5
\begin{lstlisting}[language=bash]
[sms](*\#*) (*\install*) mvapich2-gnu12-ohpc
\end{lstlisting}
% ohpc_indent 0
% ohpc_command fi
% end_ohpc_run

Alternatively, if your system includes \IntelR{} \OmniPath{}, use the (\texttt{psm2})
variant of MVAPICH2 instead:

% begin_ohpc_run
% ohpc_command if [[ ${enable_opa} -eq 1 ]];then
% ohpc_indent 5
\begin{lstlisting}[language=bash]
[sms](*\#*) (*\install*) mvapich2-psm2-gnu12-ohpc
\end{lstlisting}
% ohpc_indent 0
% ohpc_command fi
% end_ohpc_run

%%--
%% https://github.com/openhpc/ohpc/issues/1273
%% disabling until we can get pmix/openmpi/slurm to play nicely
%%--
%% An additional OpenMPI build variant is listed in Table~\ref{table:mpi} which
%% enables \href{https://pmix.github.io/pmix/}{\color{blue}{PMIx}} job launch
%% support for use with \SLURM{}. This optional variant is
%% available as \texttt{openmpi4-pmix-slurm-gnu9-ohpc}.


\subsection{Performance Tools} \label{sec:install_perf_tools}
\input{common/perf_tools}

\subsection{Setup default development environment}
\input{common/default_dev}

%\vspace*{0.2cm}
\subsection{3rd Party Libraries and Tools} \label{sec:3rdparty}
\input{common/third_party_libs_intro}

\input{common/third_party_libs}
\input{common/third_party_mpi_libs_x86}

\vspace*{.6cm}
\subsection{Optional Development Tool Builds} \label{sec:3rdparty_intel}
\input{common/oneapi_enabled_builds}

\section{Resource Manager Startup} \label{sec:rms_startup}
\input{common/slurm_startup_stateful}

\section{Run a Test Job} \label{sec:test_job}
\input{common/xcat_slurm_test_job}

\clearpage
\appendix
{\bf \LARGE \centerline{Appendices}} \vspace*{0.2cm}

\addcontentsline{toc}{section}{Appendices}
\renewcommand{\thesubsection}{\Alph{subsection}}

\input{common/automation_appendix}
\input{common/upgrade_stateful}
\input{common/test_suite}
\input{common/customization_appendix_centos}
\input{manifest}
\input{common/signature}


\end{document}

